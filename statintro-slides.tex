\documentclass{article}
\pagestyle{empty}
%\usepackage{pdfpages}
\usepackage{graphicx,xcolor}
\usepackage{geometry}
\geometry{total={180mm,150mm},
 left=20mm,
 top=20mm}
\usepackage{fontspec}
\setmainfont[Ligatures=TeX,Scale=0.95,Numbers=OldStyle]{Iwona}
\def \PDFCurrentFile{stat_w_03.pdf}
\newenvironment{slide}[1]{%%%
  {\color{green}\vbox to0pt{\noindent \vrule height 140mm width1\textwidth\vss}}
  \vspace{3mm}%% <--topsep
\noindent%
\begin{minipage}{.9\textwidth}
\includegraphics[page=#1,clip,trim=0 0 0 0]{\PDFCurrentFile}
\end{minipage}%
\par
{\color{orange}\vbox to0pt{\noindent \vrule height 40mm width1.0\textwidth\vss}}
\begin{minipage}{.9\textwidth}
  \thepage::\par
}{\end{minipage}\par
\clearpage}
%%%%%%%%%%%
\newenvironment{xslide}[2]{%%%
  {\color{green}\vbox to0pt{\noindent \vrule height 140mm width1\textwidth\vss}}
  \vspace{3mm}%% <--topsep
\noindent%
\begin{minipage}{.9\textwidth}#1
\end{minipage}%
\par
{\color{orange}\vbox to0pt{\noindent \vrule height 40mm width1.0\textwidth\vss}}
\begin{minipage}{.9\textwidth}#2
  \thepage::\par
}{\end{minipage}\par
\clearpage}
%%%%%%%%%%%%
\definecolor{orange}{rgb}{.80,0.84,0.73}
\definecolor{green}{rgb}{.80,0.84,0.13}
\begin{document}

%%\includepdf[pages=21]{stat_w_03.pdf}
%%\includegraphics[page=21]{stat_w_03.pdf}


\begin{xslide}{
    {\color{white}\vbox to0pt{\noindent \vrule height 140mm width.8\textwidth\vss}}
    \begin{minipage}{.8\textwidth}
      \vspace{30mm}
      \begin{center}
        \begin{Huge}
      Wprowadzenie do statystyki\\
      \large cz.~1\\[1mm]
      \Large
      Podstawowe pojęcia i opis statystyczny
        \end{Huge}
      \end{center}
      \vspace{44mm}
    \end{minipage}
  }{}
\end{xslide}  


\def \PDFCurrentFile{stat_w_03.pdf}



\begin{slide}{2}
\textbf{Jednostka statystyczna}: jednostki statystyczne w danej populacji różnią się od innych jednostek
spoza danej populacji poprzez swoje własności wspólne (cechy stałe), 
jednocześnie różnią się między sobą cechami (cechy zmienne)

\textbf{Cechy statystyczne} -- właściwości jednostek statystycznych
Cechy stałe -- jednakowe dla wszystkich jednostek badania: rzeczowa
(co? kto? jest badane/y) przestrzenna (gdzie?)  czasowa (kiedy?)

Cecha ilościowa (quantitative), np. numer buta, wiek;
Cecha jakościowa (qualitative), np. płeć

\end{slide}

\begin{slide}{3}
Four measurement scales: nominal (nominalna), ordinal (porządkowa), interval (interwałowa) and ratio (ilorazowa)

Skale interwałowe nie nie mają \textbf{prawdziwego zera} (brak cechy), co powoduje subtelny problem przy mnożeniu.

Przykładowo temperatura w~skali Celsjusza jest skalą interwałową: $2 \times 0 = 0$C (a~powinno być 2 razy cieplej niż przy temperaturze $0$C;
W skali Kelvina tak jest 2 $\times$ 273,15 = 546,3 ale dla zera bezwzględego $2 \times 0 = 0$K się zgadza z tym czego należało oczekiwać,
tj. że mnożenie przez $0$ daje $0$/brak cechy)

\end{slide}

\begin{slide}{4}

Główny Urząd Statystyczny (GUS) – 
podległy Prezesowi RM urząd 
zajmujący się zbieraniem/udostępnianiem informacji statystycznych na temat różnych
dziedzin życia publicznego/prywatnego.

W~2016 wydatki GUS wyniosły 409,7 mln PLN. \textbf{Średnie zatrudnienie} (co to?)
w przeliczeniu na pełne etaty wyniosło 5834 osoby.

EUROSTAT -- urząd zajmujący się sporządzaniem prognoz/analiz statystycznych dot.~obszaru UE/EFTA,
i~koordynowaniem/monitorowaniem prac narodowych urzędów statystycznych
(unifikacja metod badań/klasyfikacji)
  
\end{slide}

\begin{slide}{5}
Dane statystyczne możemy w ogólności podzielić na \textbf{dane
  przekrojowe} (cross-sectional data) -- wiele jednostek obserwowanych
w~jednym momencie/okresie czasu; \textbf{szeregi czasowe}
(time-series) -- jedna jednostka obserwowana w~wielu
momentach/okresach czasu; \textbf{dane panelowe} (panel data,
cross-sectional time-series data) -- wiele jednostek obserwowanych
w~wielu m/o czasu.

Mówiąc inaczej:
\textbf{Szeregi czasowe}: dane oznaczone stemplem czasu;
\textbf{Dane przestrzenne} (spatial): dane oznaczone pozycją na powierzchni ziemi.
  
\end{slide}

\begin{slide}{6}
  Individual/Discrete/Continuous (Data) Series;
  
  W innym aspekcie używa się \textbf{individual} vs \textbf{organizational} data, co oznacza dane na poziomie
  indywidualnym albo na poziomie organizacji (przedsiębiorstwa)
\end{slide}

\begin{slide}{7}
  Discrete (Data) Series

  Rozkład częstości (rozkład empiryczny zmiennej, szereg rozdzielczy):
  przyporządkowanie kolejnym wartościom zmiennej (xi) odpowiadających
  im liczebności (ni) lub udziałów.  Przedstawia \textbf{strukturę
  zbiorowości} dla określonej cechy (stąd analiza struktury);
  \textbf{Frequency table/distribution} vs \textbf{Relative frequency table/distribution}
    
\textbf{Tablica statystyczna}: Część liczbowa + część opisowa: tytuł; boczek (nazwy
wierszy); główka (n.~kolumn); źródło danych; ewentualne uwagi/objaśnienia.

\end{slide}

\begin{slide}{8}
Continuous (Data) Series

\textbf{Zasady grupowania danych}:
1)~Równe rozpiętości przedziałów;
2)~Niezerowe liczebności wszystkich przedziałów;
3)~Zdefiniowane wszystkie końce przedziałów;
4)~Niedominująca liczebność przedziału;
5)~,,Dobrze wyglądające'' końce przedziałów  (kończące się na zero/pięć/liczbą całkowitą)

Liczba przedziałów: określona konwencjami w~domenie zastosowań/celem badania. Raczej nie mniej niż 6--8.
(In case of doubt copy good reference.)

\end{slide}

\begin{slide}{10}
   cumulative frequency (table)

   empirical \textbf{distribution function}
\end{slide}
 
\begin{slide}{11}
\end{slide}

\begin{slide}{12}
  Mean (or Average); Median; Mode

  Dispersion: variance, standard deviation, range (rozstęp)

  Skewness (positive/negative)

  Concentration
  
\end{slide}

\begin{slide}{13}
\end{slide}

\begin{slide}{14}
\end{slide}

\begin{slide}{15}
\end{slide}

\begin{slide}{16}
\end{slide}

\begin{slide}{17}
\end{slide}

\begin{slide}{18}
\end{slide}

\begin{slide}{19}
\end{slide}

\begin{slide}{20}
Variance
\end{slide}

\begin{slide}{21}
  \textbf{Odchylenie standardowe} (\emph{standard deviation});
  \textbf{Odchylenie przeciętne} (\emph{average absolute deviation});
  Rozstęp ćwiartkowy, rozstęp międzykwartylowy (interquartile range (IQR), midspread);
  Odchylenie ćwiartkowe
  Odchylenie ćwiartkowe (\emph{Quartile coefficient of dispersion\/})

\end{slide}

\begin{slide}{22}
\end{slide}

\begin{slide}{23}
\end{slide}

\begin{slide}{24}
\end{slide}

\begin{slide}{25}
 \emph{Skewness\/}
\end{slide}

\begin{slide}{26}
  \noindent \kern8mm
  \includegraphics[width=11.5cm]{stonoga_vs_pis.png}
\end{slide}

%%\begin{slide}{27}
%%\end{slide}

\begin{slide}{28}
Herfindahl-Hirschman Index (HHI): commonly accepted measure of market
concentration. It is calculated by squaring the market share of each
firm competing in a market, and then summing the resulting numbers,
and can range from close to zero to 10,000. The U.S. Department of
Justice uses the HHI for evaluating potential mergers issues.
(cf https://www.investopedia.com/terms/h/hhi.asp)

Przykład: na rynku udziały 20 podmiotów są następujące:

$F_1 = 40\%, F_2 = 30\%, F_3 = 14\%, F_4$--$F_20 = 1\%$ każda;

HHI = $40^2 + 30^2 + \dots 1^2$ = = 1,600 + 900 + 196 + 20 = 2,716
\end{slide}

\begin{slide}{29}
\end{slide}

%%\begin{slide}{30}
%%\end{slide}


\def \PDFCurrentFile{stat_w_05.pdf}

\begin{slide}{2}
\end{slide}

\begin{slide}{3}
\end{slide}

\begin{slide}{4}
\end{slide}

%%\begin{slide}{5}
%%\end{slide}

\begin{slide}{14}
\end{slide}

\begin{slide}{15}
\end{slide}

\begin{slide}{16}
\end{slide}

\begin{slide}{17}
\end{slide}

\begin{slide}{18}
\end{slide}

\begin{slide}{19}
\end{slide}

\begin{slide}{20}
\end{slide}

\begin{slide}{21}
\end{slide}

\begin{slide}{22}
\end{slide}

\begin{slide}{23}
\end{slide}

\begin{slide}{24}
\end{slide}

\begin{slide}{25}
\end{slide}

\begin{slide}{26}
\end{slide}

\begin{xslide}{
    {\color{white}\vbox to0pt{\noindent \vrule height 140mm width.8\textwidth\vss}}
    \begin{minipage}{.8\textwidth}
  \subsection*{Wykresy}
  \begin{raggedright}
  \textbf{Dane jakościowe}: struktura (kołowy, słupkowy/barplot)

  Wysokości słupków są równe odpowiednim liczebnościom (lub częstościom); szerokość słupków jest jednakowa, zalecane jest uporządkowanie

  \textbf{Wykres kołowy}: Mało czytelne, gdy występuje więcej kategorii;
  Porównanie dwóch wykresów jest trudniejsze niż dla wykresów słupkowych.

  \medskip
  
  \textbf{Dane przekrojowe}: rozkład (słupkowy zwany także diagramem liczebości/częstości dla szeregu rozdzielczego jednostopniowego,
  histogram dla sz.r. wielostopniowego)

  \medskip
  
  \textbf{Szeregi czasowe}: dynamika (słupkowy, liniowy)

  \medskip
  
  \textbf{Porównanie}: struktury, rozkładów, dynamiki (słupkowy, pudełkowy dla danych przekrojowych)

  \medskip
  
  \textbf{Wykres pudełkowy (boxplot)}: 5 podstawowych wskaźników
  sumarycznych na jednym wykresie: $Q_1$ dolna krawędź, $Q_3$ górna
  krawędź, $Me$ środek pudełka; wąsy to maksimum/minimum (lub $\pm$1,5
  $\times$ IQR)
  
  \textbf{Dwie zmienne}: wykresy rozrzutu (scatterplots)
  %% http://quantup.pl/wprowadzenie-do-statystyki/chapter-analiza-opisowa.html
  \end{raggedright}
  \vspace{22mm}
  \end{minipage}
  }{}
\end{xslide}

\begin{xslide}{
    {\color{white}\vbox to0pt{\noindent \vrule height 140mm width.8\textwidth\vss}}
    \begin{minipage}{.8\textwidth}
  \subsection*{Wykresy}

  \includegraphics[width=5.5cm]{Nobel-Age-BoxPlot.pdf}
  \includegraphics[width=5.5cm]{komisje_glosy_razem.png}

  \kern-4mm
  
  \includegraphics[width=6cm]{pie_vs_bar.png}
    \includegraphics[width=6cm]{pies_vs_bars_m.png}
  
  \end{minipage}
  }{}
\end{xslide}  
  
\end{document}

