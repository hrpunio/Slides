%& --translate-file=il2-pl
%
% http://www.ip.ethz.ch/education/techpolicy_seminar
\documentclass[a4,portrait]{seminar}
\usepackage{iwona,polski}
\usepackage{ragged2e,array}
\RequirePackage{graphics,color,ifpdf}
%%
\slidewidth=31.0cm  %% szeroko�� slajdu
\slideheight=25.0cm %% wysoko�� slajdu
%%%
\renewcommand{\slideleftmargin}{6mm}
\renewcommand{\sliderightmargin}{6mm}
\renewcommand{\slidetopmargin}{6mm} %%%
\renewcommand{\slidebottommargin}{1mm}
\renewcommand{\slidefuzz}{9pt}
%%
\renewcommand\labelitemi{\TO}
\setlength{\itemsep}{0pt}
%%
%%\slidewidth=22.0cm \slideheight=17.0cm % do ew. zmiany ==> landscape
\slidesmag{4}
\definecolor{LightGreen}{rgb}{0.78,0.90,0.76}
\definecolor{javadocblue}{rgb}{0,.5,.1}
\definecolor{lightblue}{rgb}{0,.5,.7}
\definecolor{darkgreen}{rgb}{0,0.6,0}
\definecolor{xxlightblue}{rgb}{0.87,0.91,0.82}%%{.8,.9,.9}
%\definecolor{LightGreen}{rgb}{0.78,0.90,0.76}
\def\?{\phantom0}
\definecolor{javadocblue}{rgb}{.1,.1,.5}
\definecolor{lightblue}{rgb}{0,.5,.7}
\definecolor{darkgreen}{rgb}{0,0.6,0}
\let\OriSection\section \let\OriSubSection\subsection
\def\section#1{\OriSection{\color{javadocblue}{#1}}}
\def\subsection#1{\OriSubSection{\color{lightblue}{#1}}}
\pagecolor{LightGreen} % t�o
\slideframewidth=.4pt
\ifpdf % to fix problems making landscape seminar pdfs
\pdfpagewidth=330truemm
%%%\pdfpagewidth=210truemm
\pdfpageheight=297truemm
%%%\pdfpageheight=330truemm
\pdfhorigin=3.4truein
\pdfvorigin=1truein
\fi
\setcounter{secnumdepth}{-1} % na wszelki wypadek
\raggedbottom
\raggedright
%
\pagestyle{empty}
\newcommand{\source}[1]{\par \begin{small}#1\end{small}}
\newcommand{\heading}[1]{%
  \begin{center}
    \large\bf #1%
  \end{center}
  \vspace{1ex minus 1ex}}
%%
%%
\def\Blstress#1{\textbf{\color{javadocblue}#1}}
\def\Clstress#1{\textbf{\color{darkgreen}#1}}
\newcommand\acro[1]{\textsc{#1}}
\newcommand{\SITI}{\acro{SI/TI}}
\def\xfrac#1#2{#1\big/#2}
\newcommand{\cov}{\textrm{cov}}
\newcommand{\var}{\textrm{var}}
\newcommand\factor[1]{\index{#1}\acro{#1}} %%%
\newcommand{\FFx}{\acro{FF}}
\newcommand\semindex[1]{\index{#1}\acro{#1}}
\usepackage{graphicx}
\usepackage{url}
\raggedright
%%\setcounter{secnumdepth}{-1}
%%
%\def\TO{{\tt ->}}
\def\TO{$\to$}
\def\Stress#1{\textbf{#1}}
\def\strong#1{\textbf{#1}}\let\Strong \strong
\def\STRONG#1{\textbf{#1}}
\def\stress#1{\emph{#1}}
\title{MODELOWANIE SATYSFAKCJI, U�YTECZNO�CI\\ I~WYKORZYSTANIA OPROGRAMOWANIA\\ OPEN SOURCE}
\author{Tomasz Przechlewski}
\date{}
%\institution{Uniwersytet Gda�ski}
%\email{tomasz@gnu.univ.gda.pl}
\begin{document}
%%%%%%%%%%%%%%%%%%%%%%%%%%%%%%%%%%%%%%%%%%%%%%%%%%%%%%%%%%%%%%%%%%%%%%%%%%%%%
%\maketitle
%%%%%%%%%%%
\begin{slide}
%%%%%%%%%%%%%
\begin{minipage}{.34\hsize}

  %\hbox to\textwidth{\hss\includegraphics[width=1\hsize]{kakovin}}%
\begin{flushright}
  \hbox{\kern10mm\includegraphics[width=1\hsize]{kakovin}\kern10mm}
  {\tiny
  \hbox{\texttt{\kern10mm https://www.flickr.com/photos/tprzechlewski/2884966649/}}}%
\end{flushright}

\end{minipage}
%%%%%%%%%%%%%%
%%\kern40mm
%%%%%%
\begin{minipage}{.5\hsize}
\vbox to50mm{\vss
\begin{flushright}\bf
  %%
  Augmenting photos with metadata\\ (using exiftool)
%%
\par \vskip4mm
\colorbox{xxlightblue}{Tomasz Przechlewski}\\
\end{flushright}
}
%%
\end{minipage}
\end{slide}

\begin{slide}
  \subsection{Metadata}

  Information about information!
  
  Microsoft Office enables augmenting documents with metadata.
  
  Metadata can be saved in HTML documents in meta/title tags 

  \begin{small}
\begin{verbatim}
<meta name="Keywords" 
  content="biblioteka, biblioteka cyfrowa,prawa autorskie,biblioteka 2.0,dlibra">
<meta name="Description" 
  content="Wprowadzenie do tematu bibliote cyfrowych.">
<title>Biblioteka 2.0 / Semantyczna biblioteka</title>
...
\end{verbatim}
\end{small}
  
  Simple metadata has limited usefulness because: 

  One can describe only documents (not abstract objects), which are accessible (one has modification premission)
  
  Metadata = bag of words (ambiguity) 
  
  Metadata requirements: Open (standards);
  Extensible (Support custom schemas);
  Universal (Support diverse data/formats);
  Robust (Keep metadata and asset in sync!)
\end{slide}

\begin{slide}
  \subsection{The Smart Data}

  The Smart Data Continuum and the Road to Information Independence
  
  \begin{center}
  \includegraphics[bb=110 90 600 375,clip,page=5,width=.8\textwidth]{../PDF_workshop_XMP_Chase_Adobe.pdf}
  \end{center}

  Source: \emph{The Semantic Web: A Guide to the Future of XML, Web Services, and Knowledge Management},
  Wiley Technology Publishing, June 2003 (Ed Chase, \emph{PDF and Metadata}, Adobe Systems, Inc. 2004.)
  
\end{slide}


\begin{slide}
  \subsection{Layers of Wisdom}

  The Semantic Spectrum: Layers of Wisdom

  \begin{center}
      \includegraphics[bb=50 40 600 440,clip,page=6,width=.725\textwidth]{../PDF_workshop_XMP_Chase_Adobe.pdf}
    \end{center}

  Source: Ed Chase, \emph{PDF and Metadata}, Adobe Systems, Inc. 2004. 

\end{slide}

\begin{slide}
  \subsection{Resource Description Framework (RDF)}
  
  It is possible to describe not only documents (but abstract objects as well)
  
  Extensible

  Serialized (written/saved) as XML 

  \colorbox{xxlightblue}{\vbox{
      Resources
      
      A resource is an object described which does not have to be a~document, it can be any object 
      (physically existing or abstract)}}

Resources can be identified by a URI (or a string) 

Universal Resource Identifier (URI): 

\begin{verbatim}
http://purl.org/dc/elements/1.1/creator
\end{verbatim}

Namespace 

\begin{verbatim}
http://purl.org/dc/elements/1.1/
\end{verbatim}

Local name 

\begin{verbatim}
creator
\end{verbatim}

Name space identifier 

\begin{verbatim}
dc
\end{verbatim} 

Shorthand 

\begin{verbatim}
dc:creator
\end{verbatim}

\end{slide}


\begin{slide}
  \subsection{Resource Description Framework cont.}

  The metadata is expressed as simple expressions (sentences):
  
  \emph{Subject\/} + \emph{Property\/} + \emph{Object\/}
  
  Subject / Property = resource (ie something identified by the URI) 

  Object = resource or string

  \includegraphics[width=.75\textwidth]{FiodorRDF.pdf}

  RDF does not have any semantics. Common schemas (Dublin Core, FOAF) are needed.

  RDF schema: namespace + set of resources (including properties) + resource interpretation 
\end{slide}

\begin{slide}
\subsection{Common schemas}
  
\textbf{Dublin Core Metadata Initative}
  
Namespace URI http://purl.org/dc/elements/1.1/
  
Elements  (15): Title, Creator, Description etc...

  \textbf{Exif}

  Used by manufacturers of digital cameras. Exif metadata includes:
  image description (resolution, size, time of exposure)
  camera details (manufacturer/model, software version); camera settings 
  (aperture, time, mode, focal length, etc.);
  bibliographical description (copyright, description of an image, author/creator); geolocation data 

\begin{verbatim}  
exiftool -all FILE.jpg
\end{verbatim} 

\textbf{XMP} (ISO 16684-1:2012)

Standard based on RDF. XMP can include DC/Exif and several other
schemes of metadata.
Extensible with custom schemes (with all consequences :-).

\colorbox{xxlightblue}{\vbox{
XMP was designed to be easily extensible by the addition 
of custom namespaces. If your metadata needs are not
already covered by the core namespaces,
you can define and use your own namespaces
}}

\end{slide}

\begin{slide}
  \subsection{Metadata models}

\textbf{Controlled vocabulary}: No relationships between concepts (objects), Uncontrolled vocabulary: 
folksonomy, tags

\textbf{Taxonomy}: a controlled vocabulary with hyponomy/hyperonimy
relationships (broader/narrower) parent-child

\textbf{Thesaurus} -- is a~taxonomy that also has some `other' relationships,
such as associations

  \colorbox{xxlightblue}{\vbox{
  The associative relationship is a relationship between two concepts
  which do not belong to the same hierarchical structure, although
  they have semantic or contextual similarities RT = related term
  
  Other useful relations: equivalent/synonim (auto=car); foreign
  languages equvalents}}

  \colorbox{xxlightblue}{\vbox{
  \textbf{Ontology} (informal definition)
  An ontology is a model of (a relevant part of) the world, listing the types of object, 
  the relationships that connect them, and constraints on the ways that objects 
  and relationships can be combined.
}}
\end{slide}

\begin{slide}
  \subsection{Flickr }
  
  \vbox to0pt{\kern-9mm
    \begin{flushright}
      \includegraphics[width=.6\textwidth]{MemberSince2006.png}
      \end{flushright}
\vss
  }

  \begin{center}
  \includegraphics[width=.35\textwidth]{mediolan_zdjecie_4.png}
  \includegraphics[width=.50\textwidth]{mediolan_zdjecie_3.png}
  \end{center}

\end{slide}

\begin{slide}
\parskip0pt
\advance\baselineskip-2pt
  \subsection{Descriptions / The purpose}

  \textbf{To facilitate further search for a~photo}
  
  Example: I~remember that some time ago a~rugby tournament took place in Poland. The exact
  year is not remembered. I know it was a U-18 or U-19 junior championship. 
  I'm looking for a~photo of a~Georgian rugby player whose name I can not remember of course. 
  Adding the appropriate tags to the image -- the more the better -- will help to find what one is looking for:

  \begin{flushleft}
    \texttt{sopot rugby gruzja u18}

    or
    
    \texttt{sopot rugby gruzja u19}
  \end{flushleft}

  \colorbox{xxlightblue}{\vbox{EXIF: UserComment (EXIF lacks something more adequate), comma separated tags}}
  
  \bigskip

  \textbf{To explicity describe authorship and legal status (license)}

My photos are available under CC license. By adding relevant information the photo is not 
anonymous and the license for its dissemination is explicit: 

  \colorbox{xxlightblue}{\vbox{
  EXIF: \texttt{ Artist} and \texttt{Author} (creator credentials),
     \texttt{Copyright} (for example CC license)}}
  
  \bigskip
  
  \textbf{To make impression on family and friends}

  Geotagged photos can be shown on maps 

  \colorbox{xxlightblue}{\vbox{
      EXIF: \texttt{GPSLongitudeRef}, \texttt{GPSLongitude}, \texttt{GPSLatitudeRef},
      \texttt{GPSLatitude} and \texttt{GPSAreaInformation}}}

 \end{slide}

\begin{slide}
\parskip0pt
\advance\baselineskip-2pt
  \subsection{Why Emacs }

  Descriptions have to be added quickly (because there are a lot of~photos).
  
Filling a form / dialog box (standard in GUI applications) is too time consuming. 

Solution: an advanced programming editor (Emacs) providing: copy, paste, keyboard shortcuts, blocks, etc...
  
  
  \vbox to 0mm{
    \hbox to \textwidth{\kern10mm%%
    \includegraphics[width=.4\textwidth]{mediolan_zdjecie_10.png}}
    \vss}
\begin{center}
  \includegraphics[width=.35\textwidth]{mediolan_zdjecie_7.png}
  \includegraphics[width=.50\textwidth]{mediolan_zdjecie_9.png}
  \end{center}
\end{slide}

\begin{slide}
\advance\baselineskip-2pt
  \subsection{What is added}

Exiftool allows to view and/or modify metadata stored in JPG files
  
\colorbox{xxlightblue}{\vbox{
    \begin{tt}
    \#\# list all taga (-a) as RDF (-X)\\
    exiftool -a -X IMG20170304094013.jpg \\
    \#\# list all exif tags\\
    exiftool -EXIF:* IMG20170304094013.jpg
\end{tt}
}}

\begin{verbatim}
  xmlns:IFD0='http://ns.exiftool.ca/EXIF/IFD0/1.0/
  xmlns:ExifIFD='http://ns.exiftool.ca/EXIF/ExifIFD/1.0/
  xmlns:XMP-pdf='http://ns.exiftool.ca/XMP/XMP-pdf/1.0/
  xmlns:GPS='http://ns.exiftool.ca/EXIF/GPS/1.0/

  <IFD0:Artist>Creator: Tomasz Przechlewski</IFD0:Artist>
  <XMP-pdf:Author>Creator: Tomasz Przechlewski</XMP-pdf:Author>
  <IFD0:Copyright>Licence: CC Attribution</IFD0:Copyright>
  <ExifIFD:UserComment>test, selfportrait, cycling [http://www.flickr.com/photos/tprzechlewski/33248495615/]</ExifIFD:UserComment>
  <ExifIFD:ImageUniqueID>http://www.flickr.com/photos/tprzechlewski/33248495615/</ExifIFD:ImageUniqueID>
  <GPS:*> ... </GPS:*> 

\end{verbatim}

\colorbox{xxlightblue}{\vbox{
\begin{tt}
%\begin{verbatim}
\# Add new tag values (-TagName='Value')\\
  exiftool  -ImageUniqueID='\emph{flickrURL}' -UserComment='\emph{flickrTags}' -Artist='\emph{tp}'
  -Author='\emph{tp}' -Copyright='Licence: CC Attribution' \\
  -ImageDescription='\emph{Opis (if any)}' 
%\#exiftool -GPSLongitudeRef=E -GPSLongitude=18.559 \\
%\#   -GPSLatitudeRef=N -GPSLatitude=54.439 -GPSAreaInformation='Sopot\#Molo-poczatek'
\end{tt}
%\end{verbatim}
%#exiftool -GPSLongitudeRef=E -GPSLongitude=18.559
%#   -GPSLatitudeRef=N -GPSLatitude=54.439 -GPSAreaInformation='Sopot#Molo-poczatek'
}}
\end{slide}

\begin{slide}
  \advance\baselineskip-2pt
  \subsection{What is added cont. (geolocations)}

  \colorbox{xxlightblue}{\vbox{
\begin{tt}
  %\begin{verbatim}
exiftool -GPSLongitudeRef=E -GPSLongitude=18.559
   -GPSLatitudeRef=N -GPSLatitude=54.439 -GPSAreaInformation='Sopot\#Molo-poczatek'
%\end{verbatim}
\end{tt}
}}

\subsection{Exiftool -- examples}

  \begin{small}
\begin{verbatim}
#Writes Artist tag to a|b|c.jpg (or directory ./images). Since no group is specified,
#EXIF:Artist will be written and all other existing Artist tags
exiftool -artist="Phil Harvey" -copyright="2011 Phil Harvey" a.jpg b.jpg c.jpg
exiftool -artist="Phil Harvey" -copyright="2011 Phil Harvey" ./images

# Writes ImageUniqueID|UserComment|Artist|Author|Copyright|ImageDescription to file
#exiftool  -ImageUniqueID='flickrURL' -UserComment='tags' -Artist='tp'
#  -Author='tp' -Copyright -ImageDescription='Opis' file.jpg
#
# Writes geocoordinates to file
#exiftool -GPSLongitudeRef=E -GPSLongitude=139.7513889
#   -GPSLatitudeRef=N -GPSLatitude=35.685 -GPSAreaInformation='Sopot' file.jpg

# Extract Camera Model
exiftool -Model IMG_20170304_161946.jpg
Camera Model Name  : Redmi 3S

#Extract all author-related XMP information from an image.
exiftool -xmp:author:all -a image.jpg
\end{verbatim}
  \end{small}

  \end{slide}


\begin{slide}
  
\subsection{Exiftool -- examples cont.}

\begin{small}
\begin{verbatim}
#Copy all tags from  src.jpg to dst.jpg
exiftool -tagsfromfile src.jpg dst.jpg

# Erase all meta information from dst.jpg image, then copy EXIF tags from src.jpg.
exiftool -all= -tagsfromfile src.jpg -exif:all dst.jpg

# Add (+=) 32 seconds to all times 
# in all files from ./Images directory
exiftool -alldates+=00:00:32 ./Images

# Remove all gps-related tags
exiftool -gps:all= *.jpg
\end{verbatim}

%% # Create KML from set of images
%% #!/bin/bash
%% # Create KML from geotagged photos
%% FMTDIR=$HOME/.config/exiftool
%% cat $FMTDIR/kml-header.fmt > out.kml

%% # -n -r -q numerical tags/resursive/quiet
%% # -p FMTFILE = Print output in specified format
%% exiftool -n -r -q -p $FMTDIR/kml-placemark.fmt . >> out.kml

%% cat $FMTDIR/kml-trailer.fmt >> out.kml

\end{small}

  
%  \begin{center}
%    %\includegraphics[bb=1 1 500 425,clip,width=.400\textwidth]{mediolan_zdjecie_15.png}
%    %\includegraphics[bb=1 1 500 300,clip,width=.570\textwidth]{mediolan_zdjecie_14.png}
%    \includegraphics[bb=1 1 500 345,clip,width=.400\textwidth]{mediolan_zdjecie_15.png}
%    \includegraphics[bb=1 1 500 242,clip,width=.570\textwidth]{mediolan_zdjecie_14.png}
%  \end{center}

%%\begin{small}
%%  Syncing photos with GPS receiver log (GPX format):
%% 
%%\begin{verbatim}
%%# rename filename to unique ones
%%My_rename_photo_files.sh
%%# defining camera time offset:
%%My-exif-datetime -b gps_display_time -f picture -o GMT_offset
%%
%%# adding GPS data from FILE.gpx (-f FILE.gpx)
%%My-photo-sync.sh -f ~/GPS/20160212.gpx -a
%%#gpsPhoto.pl --dir=DIR --gpsfile=FILE --timeoffset=OFFSET --overwrite-geotagged --kml NAME.kml
%%
%%# Conversion to KML (for google maps/OSM import)
%%mk_kml.pl -gpx FILE.gpx -log FILE.log -name 'KML File Title/Name' > FILE.kml
%%\end{verbatim}
%%\end{small}

%% \end{slide}

\vbox to0pt{\kern-33mm
  \begin{flushright}
    \includegraphics[width=.55\textwidth]{exif-pilot-screen.png}
  \end{flushright}
  \vss
}


\subsection{GUI applications}

Add/modify metadata with Exif pilot.


\end{slide}


\begin{slide}
  \subsection{Google mymaps}

  The KML file created by \texttt{mk\_kml.pl} can be displayed on GoogleMaps or similar. 

  Photos are stored at \texttt{flickr.com}.

  Can be stored/downloaded from any server, our own is the best of course, as using some third-party (free)
  facility one can experience so-called \textbf{disappearing cloud (provider) problem}.

    \colorbox{xxlightblue}{\vbox{
\begin{tt}
  https://www.google.pl/maps/d/
    \end{tt}}}
    
  \begin{center}
    \includegraphics[width=.41\textwidth]{MediolanBergamo0.png}
    \includegraphics[width=.555\textwidth]{MediolanBergamo1.png}
\end{center}
\end{slide}

\begin{slide}
  \subsection{umap.fr}

  Displaying KML at \texttt{umap.fr} (instead of GoogleMaps)
  
  \colorbox{xxlightblue}{\vbox{
\begin{tt}
%%
%%\begin{verbatim}
https://umap.openstreetmap.fr/en/map/italy2016{\textunderscore}71547\#13/45.4687/9.1960
%%\end{verbatim}
  \end{tt}}}
  
\begin{center}
\includegraphics[width=.5\textwidth]{BergamoMediolan5.png}
\end{center}
\end{slide}


\begin{slide}
  \subsection{KML (google maps)}

\begin{small}
\begin{verbatim}
<Placemark><name>IMG_20170304_094013.jpg</name>
 <description><![CDATA[<a href='http://www.flickr.com/tprzechlewski/33248495615'>
  <img src='https://farm4.staticflickr.com/3852/33248495615_1537851166_m.jpg' 
    width='200' /></a><br><a href='http://www.flickr.com/tprzechlewski/33248495615'>#fullSize</a>]]>
  </description>
 <ExtendedData>
   <Data name='gx_media_links'>
     <value>https://farm4.staticflickr.com/3852/33248495615_1537851166_m.jpg</value>
   </Data>
 </ExtendedData>
 <Point>
  <coordinates>18.05169167,54.71651111</coordinates>
 </Point>
</Placemark>
\end{verbatim}  
\end{small}

\subsection{Markdown (umap.fr)}

\begin{small}
\begin{verbatim}  
<description>
{{https://farm4.staticflickr.com/3852/33248495615_1537851166_m.jpg}}
[[http://www.flickr.com/tprzechlewski/33248495615|enlarge/powieksz]]
</description>
\end{verbatim}  
\end{small}

\end{slide}

\begin{slide}
%%%%%%%%%%%%%
  \begin{minipage}{.45\hsize}
\begin{flushright}
  \hbox{\kern10mm\includegraphics[width=.8\hsize]{SopotMolo_poczatek.jpg}\kern3mm}
  {\tiny
  \hbox{\texttt{\kern10mm https://www.flickr.com/photos/tprzechlewski/91096674/}}}%
\end{flushright}
\end{minipage}
\begin{minipage}{.50\hsize}
  \vbox to 40mm{%\vss
    \raggedright
    Endnote: My flickr account is active from 2006. Proof included: Sopot pier in february/2005, uploaded to flickr in january/2006
    \vss
\begin{flushleft}\bf
\qquad The END $|$ Thanks
\end{flushleft}
\kern-2.7mm
}
%%
\end{minipage}
\end{slide}
\end{document}

%http://www.davidkennardphotography.com/blog/851-getting-organised-photo-filenames-metadata.xhtml
%google: exiftool ImageUniqueID

% Local Variables:
% TeX-master: ""
% mode: latex
% coding: iso-8859-2
% ispell-local-dictionary: "polish"
% End:

